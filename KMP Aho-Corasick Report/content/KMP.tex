\section{Knuth - Morris - Pratt Step by step}
\textbf{Note}: The string will start at number 0

\subsection{Problem}
\begin{itemize}
    \item Given string s: "ABXABABABAA".
    \item Given string t: "ABABAA".
\end{itemize}

Check if string t exists in string s?

\subsection{Main idea}

The idea behind KMP is to extend a suffix ending at position i to a suffix ending at position i+1 while still matching the corresponding prefix, which improves efficiency by eliminating expensive string comparisons. \cite{vnoi}


\subsection{Definition}
\begin{itemize}
    \item \textbf{Proper Prefix}: A prefix that is not equal to the string itself.

    \item \textbf{Prefix Function}\cite{cpAlgo}: An array $\pi$ of length m, where $\pi[i]$ is the length of the longest proper prefix of the substring $s[0..i]$, which is also a suffix of the substring. Mathematically, the definition of the prefix function can be written as follows:
\end{itemize}
\begin{equation}
    \pi[i] = \max_{0 \le k \le i} \{\,k : s[0 \dots k-1] = s[i-(k-1) \dots i]\}.
\end{equation}

\subsection{Preprocessing}
Due to the definition of proper prefix, $\pi[0] = 0$.

Here is the example code:
\lstinputlisting[language=C++]{code/Preprocessing.cpp}
The example code will run in $\Theta(m)$ where m is the length of the pattern.
    \subsection *{Explain step by step}
    \textbf{Step 1: }$\pi[0] = 0$ so $i$ starts at 1. We use $j$ to trace the prefix, and it starts at 0:

    \begin{table}[H]
    \centering
    \begin{tabular}{|c|c|c|c|c|c|c|}
    \hline
    pos   & j & i &   &   &   &   \\ \hline
    t     & A & B & A & B & A & A \\ \hline
    $\pi$ & 0 &   &   &   &   &   \\ \hline
    \end{tabular}
    \end{table}
    
    \textbf{Step 2: }\textit{At $i = 1$ and $j = 0$}
    
    $t[i]$ mismatches $t[j]$ ("A" $\neq$ "B"), so we set $\pi[i = 1] = 0$ then increases $i$ by 1:
    \begin{table}[H]
    \centering
    \begin{tabular}{|c|c|c|c|c|c|c|}
    \hline
    pos   & $j$ &   & $i$ &   &   &   \\ \hline
    t     & A & B & A & B & A & A \\ \hline
    $\pi$ & 0 & 0 &   &   &   &   \\ \hline
    \end{tabular}
    \end{table}

    \textbf{Step 3: }\textit{At $i = 2$ and $j = 0$}
    
    $t[i]$ matches $t[j]$ ("A" $=$ "A"), so we set $\pi[i = 2] = j + 1 = 1$ then increases both $i$ and $j$ by 1:
    \begin{table}[H]
    \centering
    \begin{tabular}{|c|c|c|c|c|c|c|}
    \hline
    pos   &   & $j$ &   & $i$ &   &   \\ \hline
    t     & A & B & A & B & A & A \\ \hline
    $\pi$ & 0 & 0 & 1 &   &   &   \\ \hline
    \end{tabular}
    \end{table}
    \textbf{Note}: $\pi[i] = j + 1$ because $\pi[i]$ represents the length of the longest proper prefix of the substring $t[0\dots i]$, which is also a suffix of the substring.
    
    \textbf{Step 4: }\textit{At $i = 3$ and $j = 1$}
    
    $t[i]$ matches $t[j]$ ("B" $=$ "B"), so we set $\pi[i = 3] = j + 1 = 2$ then increases both $i$ and $j$ by 1:
    \begin{table}[H]
    \centering
    \begin{tabular}{|c|c|c|c|c|c|c|}
    \hline
    pos   &   &   & $j$ &   & $i$ &   \\ \hline
    t     & A & B & A & B & A & A \\ \hline
    $\pi$ & 0 & 0 & 1 & 2 &   &   \\ \hline
    \end{tabular}
    \end{table}
    \textbf{Step 5: }\textit{At $i = 4$ and $j = 2$}
    
    $t[i]$ matches $t[j]$ ("A" $=$ "A"), so we set $\pi[i = 4] = j + 1 = 3$ then increases both $i$ and $j$ by 1:
    \begin{table}[H]
    \centering
    \begin{tabular}{|c|c|c|c|c|c|c|}
    \hline
    pos   &   &   &   & $j$ &   & $i$ \\ \hline
    t     & A & B & A & B & A & A \\ \hline
    $\pi$ & 0 & 0 & 1 & 2 & 3 &   \\ \hline
    \end{tabular}
    \end{table}
    \textbf{Step 6: }\textit{At $i = 5$ and $j = 3$} 
    
    $t[i]$ mismatches $t[j]$ ("B" $\neq$ "A") and j is greater than 0. This situation is a bit tricky because we must trace back to the $\pi[j - 1]$ position. Since $\pi[i]$ represents the longest proper prefix of $s[0\dots i]$, $\pi[i - 1]$ represents the second longest proper prefix and so on. Therefore, if there is a mismatch, we have to trace back to the second, third, etc., longest proper prefix. In this case, $j$ moves to position 1 because $\pi[j - 1] = 1$:
    \begin{table}[H]
    \centering
    \begin{tabular}{|c|c|c|c|c|c|c|}
    \hline
    pos   &   & $j$ &   &   &   & $i$ \\ \hline
    t     & A & B & A & B & A & A \\ \hline
    $\pi$ & 0 & 0 & 1 & 2 & 3 &   \\ \hline
    \end{tabular}
    \end{table}
    \textbf{Step 7: }Now, $t[i]$ and $t[j]$ still do not match, and $j$ is still greater than 0, so we repeat the backtracking step. As a result, $j$ goes back to $0$ because $\pi[j - 1] = 0$:
    \begin{table}[H]
    \centering
    \begin{tabular}{|c|c|c|c|c|c|c|}
    \hline
    pos   & $j$ &   &   &   &   & $i$ \\ \hline
    t     & A & B & A & B & A & A \\ \hline
    $\pi$ & 0 & 0 & 1 & 2 & 3 &   \\ \hline
    \end{tabular}
    \end{table}
    \textbf{Step 8: }Now, $t[i]$ and $t[j]$ match ("A" $=$ "A") so we set $\pi[i] = j + 1 = 1$ and done the prefix function.
    \begin{table}[H]
    \centering
    \begin{tabular}{|c|c|c|c|c|c|c|}
    \hline
    t     & A & B & A & B & A & A \\ \hline
    $\pi$ & 0 & 0 & 1 & 2 & 3 & 1 \\ \hline
    \end{tabular}
    \end{table}

    \textbf{Important}: The code always runs in linear time regardless of the string's distribution. Therefore, the time and space complexity of the code is $\Theta(m)$, where m is the length of the pattern string.
\subsection{String matching}

After computing the prefix function for the pattern \( t \), we use it to efficiently search for \( t \) in the text \( s \). Recall that for \( t = \texttt{ABABAA} \), the prefix function is:
\[
\pi = [0,\, 0,\, 1,\, 2,\, 3,\, 1].
\]

The string matching algorithm proceeds as follows:

\begin{enumerate}
    \item Initialize two indices: \( i \) for the text \( s \) (starting at 0) and \( j \) for the pattern \( t \) (starting at 0).
    \item Compare \( s[i] \) with \( t[j] \):
    \begin{itemize}
        \item If \( s[i] = t[j] \), increment both \( i \) and \( j \).
        \item If \( s[i] \neq t[j] \) and \( j > 0 \), update \( j \) to \( \pi[j-1] \) (i.e., backtrack \( j \) to the length of the next best candidate prefix) and compare again.
        \item If \( s[i] \neq t[j] \) and \( j = 0 \), increment \( i \) only.
    \end{itemize}
    \item If \( j \) reaches the length of \( t \) (i.e., \( j = m \) where \( m \) is the length of \( t \)), a full match is found at index \( i - j \) in \( s \).
\end{enumerate}

\subsection*{Example Walk-through}
Consider the given strings:
\begin{itemize}
    \item \( s = \texttt{ABXABABABAA} \)
    \item \( t = \texttt{ABABAA} \)
\end{itemize}
\textbf{Step 1: }\textit{Start with i = 0 and j = 0}

\begin{table}[H]
\centering
\begin{tabular}{|c|c|c|c|c|c|c|c|c|c|c|c|}
\hline
pos   &$i$&   &   &   &   &   &   &   &   &   &   \\ \hline
s     & A & B & X & A & B & A & B & A & B & A & A \\ \hline
\end{tabular}
\end{table}

\begin{table}[H]
\centering
\begin{tabular}{|c|c|c|c|c|c|c|}
\hline
pos   & $j$ &   &   &   &   &   \\ \hline
t     & A & B & A & B & A & A \\ \hline
$\pi$ & 0 & 0 & 1 & 2 & 3 & 1 \\ \hline
\end{tabular}
\end{table}

$s[0] = $ \texttt{A} and $t[0] = $ \texttt{A} $\longrightarrow$ match, so increment: $i = 1$, $j = 1$.

\textbf{Step 2: }\textit{Now, $i = 1$ and $j = 1$}

\begin{table}[H]
\centering
\begin{tabular}{|c|c|c|c|c|c|c|c|c|c|c|c|}
\hline
pos   &   &$i$&   &   &   &   &   &   &   &   &   \\ \hline
s     & A & B & X & A & B & A & B & A & B & A & A \\ \hline
\end{tabular}
\end{table}

\begin{table}[H]
\centering
\begin{tabular}{|c|c|c|c|c|c|c|}
\hline
pos   &   & $j$ &   &   &   &   \\ \hline
t     & A & B & A & B & A & A \\ \hline
$\pi$ & 0 & 0 & 1 & 2 & 3 & 1 \\ \hline
\end{tabular}
\end{table}

$s[1] = $ \texttt{B} and $t[1] = $ \texttt{B} $\longrightarrow$ match, so increment: $i = 2$, $j = 2$.

\textbf{Step 3: }\textit{Now, $i = 2$ and $j = 2$}

\begin{table}[H]
\centering
\begin{tabular}{|c|c|c|c|c|c|c|c|c|c|c|c|}
\hline
pos   &   &   &$i$&   &   &   &   &   &   &   &   \\ \hline
s     & A & B & X & A & B & A & B & A & B & A & A \\ \hline
\end{tabular}
\end{table}

\begin{table}[H]
\centering
\begin{tabular}{|c|c|c|c|c|c|c|}
\hline
pos   &   &   & $j$ &   &   &   \\ \hline
t     & A & B & A & B & A & A \\ \hline
$\pi$ & 0 & 0 & 1 & 2 & 3 & 1 \\ \hline
\end{tabular}
\end{table}

$s[2] = $ \texttt{X} and $t[2] = $ \texttt{A} $\longrightarrow$ mismatch.

Since $j > 0$, update $j$ to $\pi[j - 1] = \pi[1] = 0$.

\textbf{Step 4: }\textit{Now, $i = 2$ and $j = 0$}

\begin{table}[H]
\centering
\begin{tabular}{|c|c|c|c|c|c|c|c|c|c|c|c|}
\hline
pos   &   &   &$i$&   &   &   &   &   &   &   &   \\ \hline
s     & A & B & X & A & B & A & B & A & B & A & A \\ \hline
\end{tabular}
\end{table}

\begin{table}[H]
\centering
\begin{tabular}{|c|c|c|c|c|c|c|}
\hline
pos   & $j$ &   &   &   &   &   \\ \hline
t     & A & B & A & B & A & A \\ \hline
$\pi$ & 0 & 0 & 1 & 2 & 3 & 1 \\ \hline
\end{tabular}
\end{table}

$s[2] = $ \texttt{X} and $t[0] = $ \texttt{A} $\longrightarrow$ mismatch.

But now $j = 0$, so increment: $i = 3$.

\textbf{Step 5: }\textit{Now, $i = 3$ and $j = 0$}

\begin{table}[H]
\centering
\begin{tabular}{|c|c|c|c|c|c|c|c|c|c|c|c|}
\hline
pos   &   &   &   &$i$&   &   &   &   &   &   &   \\ \hline
s     & A & B & X & A & B & A & B & A & B & A & A \\ \hline
\end{tabular}
\end{table}

\begin{table}[H]
\centering
\begin{tabular}{|c|c|c|c|c|c|c|}
\hline
pos   & $j$ &   &   &   &   &   \\ \hline
t     & A & B & A & B & A & A \\ \hline
$\pi$ & 0 & 0 & 1 & 2 & 3 & 1 \\ \hline
\end{tabular}
\end{table}

$s[3] = $ \texttt{A} and $t[0] = $ \texttt{A} $\longrightarrow$ match, so increment: $i = 4$, $j = 1$.

\textbf{Step 6: }\textit{Now, $i = 4$ and $j = 1$}

\begin{table}[H]
\centering
\begin{tabular}{|c|c|c|c|c|c|c|c|c|c|c|c|}
\hline
pos   &   &   &   &   &$i$&   &   &   &   &   &   \\ \hline
s     & A & B & X & A & B & A & B & A & B & A & A \\ \hline
\end{tabular}
\end{table}

\begin{table}[H]
\centering
\begin{tabular}{|c|c|c|c|c|c|c|}
\hline
pos   &   & $j$ &   &   &   &   \\ \hline
t     & A & B & A & B & A & A \\ \hline
$\pi$ & 0 & 0 & 1 & 2 & 3 & 1 \\ \hline
\end{tabular}
\end{table}

$s[4] = $ \texttt{B} and $t[1] = $ \texttt{B} $\longrightarrow$ match, so increment: $i = 5$, $j = 2$.

\textbf{Step 7: }\textit{Now, $i = 5$ and $j = 2$}

\begin{table}[H]
\centering
\begin{tabular}{|c|c|c|c|c|c|c|c|c|c|c|c|}
\hline
pos   &   &   &   &   &   &$i$&   &   &   &   &   \\ \hline
s     & A & B & X & A & B & A & B & A & B & A & A \\ \hline
\end{tabular}
\end{table}

\begin{table}[H]
\centering
\begin{tabular}{|c|c|c|c|c|c|c|}
\hline
pos   &   &   &$j$&   &   &   \\ \hline
t     & A & B & A & B & A & A \\ \hline
$\pi$ & 0 & 0 & 1 & 2 & 3 & 1 \\ \hline
\end{tabular}
\end{table}

$s[5] = $ \texttt{A} and $t[2] = $ \texttt{A} $\longrightarrow$ match, so increment: $i = 6$, $j = 3$.

\textbf{Step 8: }\textit{Now, $i = 6$ and $j = 3$}

\begin{table}[H]
\centering
\begin{tabular}{|c|c|c|c|c|c|c|c|c|c|c|c|}
\hline
pos   &   &   &   &   &   &   &$i$&   &   &   &   \\ \hline
s     & A & B & X & A & B & A & B & A & B & A & A \\ \hline
\end{tabular}
\end{table}

\begin{table}[H]
\centering
\begin{tabular}{|c|c|c|c|c|c|c|}
\hline
pos   &   &   &   &$j$&   &   \\ \hline
t     & A & B & A & B & A & A \\ \hline
$\pi$ & 0 & 0 & 1 & 2 & 3 & 1 \\ \hline
\end{tabular}
\end{table}

$s[6] = $ \texttt{B} and $t[3] = $ \texttt{B} $\longrightarrow$ match, so increment: $i = 7$, $j = 4$.

\textbf{Step 9: }\textit{Now, $i = 7$ and $j = 4$}

\begin{table}[H]
\centering
\begin{tabular}{|c|c|c|c|c|c|c|c|c|c|c|c|}
\hline
pos   &   &   &   &   &   &   &   &$i$&   &   &   \\ \hline
s     & A & B & X & A & B & A & B & A & B & A & A \\ \hline
\end{tabular}
\end{table}

\begin{table}[H]
\centering
\begin{tabular}{|c|c|c|c|c|c|c|}
\hline
pos   &   &   &   &   &$j$&   \\ \hline
t     & A & B & A & B & A & A \\ \hline
$\pi$ & 0 & 0 & 1 & 2 & 3 & 1 \\ \hline
\end{tabular}
\end{table}

$s[7] = $ \texttt{A} and $t[4] = $ \texttt{A} $\longrightarrow$ match, so increment: $i = 8$, $j = 5$.

\textbf{Step 10: }\textit{Now, $i = 8$ and $j = 5$}

\begin{table}[H]
\centering
\begin{tabular}{|c|c|c|c|c|c|c|c|c|c|c|c|}
\hline
pos   &   &   &   &   &   &   &   &   &$i$&   &   \\ \hline
s     & A & B & X & A & B & A & B & A & B & A & A \\ \hline
\end{tabular}
\end{table}

\begin{table}[H]
\centering
\begin{tabular}{|c|c|c|c|c|c|c|}
\hline
pos   &   &   &   &   &   &$j$\\ \hline
t     & A & B & A & B & A & A \\ \hline
$\pi$ & 0 & 0 & 1 & 2 & 3 & 1 \\ \hline
\end{tabular}
\end{table}

$s[8] = $ \texttt{B} and $t[5] = $ \texttt{A} $\longrightarrow$ mismatch.

Since $j > 0$, update $j$ to $\pi[j - 1] = \pi[4] = 3$.

\textbf{Step 11: }\textit{Now, $i = 8$ and $j = 3$}

\begin{table}[H]
\centering
\begin{tabular}{|c|c|c|c|c|c|c|c|c|c|c|c|}
\hline
pos   &   &   &   &   &   &   &   &   &$i$&   &   \\ \hline
s     & A & B & X & A & B & A & B & A & B & A & A \\ \hline
\end{tabular}
\end{table}

\begin{table}[H]
\centering
\begin{tabular}{|c|c|c|c|c|c|c|}
\hline
pos   &   &   &   &$j$&   &   \\ \hline
t     & A & B & A & B & A & A \\ \hline
$\pi$ & 0 & 0 & 1 & 2 & 3 & 1 \\ \hline
\end{tabular}
\end{table}

$s[8] = $ \texttt{B} and $t[3] = $ \texttt{B} $\longrightarrow$ match, so increment: $i = 9$, $j = 4$.

\textbf{Step 12: }\textit{Now, $i = 9$ and $j = 4$}

\begin{table}[H]
\centering
\begin{tabular}{|c|c|c|c|c|c|c|c|c|c|c|c|}
\hline
pos   &   &   &   &   &   &   &   &   &   &$i$&   \\ \hline
s     & A & B & X & A & B & A & B & A & B & A & A \\ \hline
\end{tabular}
\end{table}

\begin{table}[H]
\centering
\begin{tabular}{|c|c|c|c|c|c|c|}
\hline
pos   &   &   &   &   &$j$&   \\ \hline
t     & A & B & A & B & A & A \\ \hline
$\pi$ & 0 & 0 & 1 & 2 & 3 & 1 \\ \hline
\end{tabular}
\end{table}

$s[9] = $ \texttt{A} and $t[4] = $ \texttt{A} $\longrightarrow$ match, so increment: $i = 10$, $j = 5$.

\textbf{Step 13: }\textit{Now, $i = 10$ and $j = 5$}

\begin{table}[H]
\centering
\begin{tabular}{|c|c|c|c|c|c|c|c|c|c|c|c|}
\hline
pos   &   &   &   &   &   &   &   &   &   &   &$i$\\ \hline
s     & A & B & X & A & B & A & B & A & B & A & A \\ \hline
\end{tabular}
\end{table}

\begin{table}[H]
\centering
\begin{tabular}{|c|c|c|c|c|c|c|}
\hline
pos   &   &   &   &   &   &$j$\\ \hline
t     & A & B & A & B & A & A \\ \hline
$\pi$ & 0 & 0 & 1 & 2 & 3 & 1 \\ \hline
\end{tabular}
\end{table}

$s[10] = $ \texttt{A} and $t[5] = $ \texttt{A} $\longrightarrow$ match.

Since $j = m$ where m is the length of t, a full match is found at index $i - j = 5^{th}$ in $s$.

Here is the full code of String Matching:
\lstinputlisting[language=C++]{code/StringMatching.cpp}

The overall complexity of this string matching process is \( \mathcal{O}(n + m) \), where \( n \) is the length of the text \( s \) and \( m \) is the length of the pattern \( t \).

\subsection{Complexity Analysis}
\begin{itemize}
    \item The total \textbf{Space Complexity} of the algorithm is $\Theta(m)$, due to the contruction of $\pi$ length m.
    \item However, the \textbf{Time Complexity} of the algorithm depends on the kind of problem:
    \begin{itemize}
        \item Find all the position that string pattern $t$ occurs in string text $s$.
        \item Return the first string pattern $t$ occurs in string text $s$ or Check if string pattern $t$ occurs in string text or not.
    \end{itemize}
\end{itemize}
\subsection*{Find all the position}
\begin{itemize}
    \item In this problem, we must agree that the \textbf{Preprocessing} step always runs in $\Theta(m)$ because it always iterates through pattern $t$.
    \item In \textbf{String Matching} step, we also iterate through string $s$ to find all the position that appears pattern $t$. So the time complexity is $\Theta(n)$.
\end{itemize}
$\Longrightarrow$ The total \textbf{Time Complexity} is $\Theta(n + m)$
\subsection*{Check if occurs or not}
\begin{itemize}
    \item In this problem, we must agree that the \textbf{Preprocessing} step always runs in $\Theta(m)$ because it always iterates through pattern $t$.
    \item In \textbf{String Matching} step, it happens 2 cases:
    \begin{itemize}
        \item \textbf{Best Case:} We instantly find the pattern $t$ at the prefix of the string $s$.
        \item \textbf{Worst Case:} We find the pattern $t$ at the suffix of the string $s$.
    \end{itemize}
\end{itemize}
$\Longrightarrow$ The \textbf{Time Complexity} of the algorithm is:
\begin{itemize}
    \item \textbf{Best Case:} $\mathcal{O}(m)$.
    \item \textbf{Worst Case: } $\mathcal{O}(n + m)$.
\end{itemize}